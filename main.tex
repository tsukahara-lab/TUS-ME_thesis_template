%%%%%%%%%%%%%%%%%%%%%%%%%%%%%%%%%%%%%%%%%%%%%%%%%%%%%%%%%%%%%%%%%%%%%%%
%%%
%%%                東京理科大学 創域理工学部 機械航空宇宙工学科
%%%                   【非公式】学位論文 LaTeX テンプレート
%%%
%%%       <https://github.com/tsukahara-lab/TUS-ME_thesis_template>
%%%
%%%                                  v1.0.0 Yuki MATSUKAWA 27 Dec. 2023
%%%                                  v2.0.0 Yuki MATSUKAWA 15 Nov. 2024
%%%                                  v2.1.0 Yuki MATSUKAWA 31 Jan. 2025
%%%
%%%%%%%%%%%%%%%%%%%%%%%%%%%%%%%%%%%%%%%%%%%%%%%%%%%%%%%%%%%%%%%%%%%%%%%

%%% 文書クラスの設定 %%%
\documentclass[
    paper=a4paper,      % A4 用紙サイズ
    report,             % report 相当の文書クラス
    fleqn,              % 数式を左寄せ
    fontsize=12pt,      % 欧文サイズ 12 pt
    jafontsize=12pt,    % 和文サイズ 12 pt
    head_space=33mm,    % 天の余白(柱とノンブルがあるので 20 mm よりも広い)
    foot_space=30mm,    % 地の余白(ノンブルが下の場合があるので 15 mm よりも広い)
    gutter=25mm,        % のどの余白
    fore-edge=10mm,     % 小口の余白
    % draft               % 画像を省略(タイプセットを高速化.提出時はコメントアウト)
    ]{jlreq}            % jlreq クラスを使用

%%% 学位論文設定ファイル %%%
\usepackage{settings}

%%% 行番号の表示 %%%
% 添削時には行番号を付けるとわかりやすい
% 提出時にはコメントアウトする
\linenumbers

%%% ここから上を「プリアンブル」と言います.パッケージや独自の設定,マクロはプリアンブルや settings.styに書いてください.
%%% ここから下が論文の本体です.
\begin{document}

%%%%%%%%%%%%%%%%
%%%%% 表紙 %%%%%
%%%%%%%%%%%%%%%%

% 卒業・修了「年度」を入力
\thesis{20**年度卒業論文}   % 卒業論文はこれ
% \thesis{20**年度修士論文}   % 修士論文はこれ

% 学位論文題目
% ここには学位論文のタイトルを入れます.一文字でも間違えたら受理されません.
% タイトルが長くて改行するときは \\ を入れる.
\title{ここには学位論文のタイトルを入れます.\\ 一文字でも間違えたら受理されません.}

% 卒業・修了「年」を入力
\date{20**年2月}

% 卒業論文の場合はこれ
% 大学名,学部名,学科名の間にスペースは不要
\affiliation{東京理科大学創域理工学部機械航空宇宙工学科}

% 修士論文の場合はこれ
% 大学名,研究科名,専攻名の間にスペースは不要
% \affiliation{東京理科大学大学院創域理工学研究科機械航空宇宙工学専攻}

% 研究室名を入力
\laboratory{〇〇研究室}

% 著者情報
\author{%
% 学籍番号を全角 7 桁で入力
75*****
\hskip2\zw% 学籍番号と氏名の間のスペース,消さない
% 姓と名の間は全角 1 文字スペース
姓姓 名名
} % 消さない

% 表紙の出力
\makecover

%%% 目次 %%%
\tableofcontents

%%% 記号表 %%%
\input{chapter/signary.tex}

%%%%%%%%%%%%%%%%
%%%%% 本文 %%%%%
%%%%%%%%%%%%%%%%
\clearpage
\pagestyle{normal}
\setcounter{page}{0}
\pagenumbering{arabic}
% 上のコマンドは消さないで.
% 本文はこれ以降に記載する.

% LaTeX ソースは一つの tex ファイルに書くのではなく,章ごとの tex ファイルに分割して書きましょう.
% 分割したファイルを読み込むときは \input{xxx} または \include{xxx} を使います.
% 以下,論文構成例を示します.自分の論文構成に合わせて書き換えてください.

%%% 序論 %%%
\input{chapter/introduction.tex}

%%% 計算手法 %%%
\input{chapter/method.tex}

%%% 結果 %%%
\input{chapter/result.tex}

%%% 考察 %%%
\input{chapter/discussion.tex}

%%% 結論 %%%
\input{chapter/conclusion.tex}

%%% 謝辞 %%%
\input{chapter/acknowledgement.tex}

%%% 文献 %%%
\biblist
% \nocite{*} が有効のとき,引用していない文献も含めて全て表示
% 確認用なので論文提出前には必ず \nocite{*} をコメントアウトすること
% \nocite{*}
% 使用する bst ファイル
\bibliographystyle{jsme}
% 読み込む bib ファイル
\bibliography{
    mybib_en.bib,
    mybib_jp.bib
}

%%% 付録 %%%
\input{chapter/appendix.tex}

%%%%%%%%%%%%%%%%%%%%%%%%%%%%%%
%%% 文章を書けるのはここまで %%%
%%%%%%%%%%%%%%%%%%%%%%%%%%%%%%
\end{document}
