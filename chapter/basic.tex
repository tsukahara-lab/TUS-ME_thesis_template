\chapter{\LaTeX の基本}
\label{ch:basic}


\section{\LaTeX での文章の書き方}
\label{sec:sentence_in_LaTeX}

\subsection{章・節・小節}
\label{ssec:ch_sec_ssec}

この学位論文テンプレートは \verb|report| と呼ばれる文書クラスを使用しているため,章(chapter),節(section),小節(subsection)に分けて文章を書けます.
例えば今読んでいるこの文章は第~\ref{ch:basic}章の第~\ref{ssec:ch_sec_ssec}節に位置しています.
それぞれの章や節のタイトルをつけるには \verb|\chapter{}|,\verb|\section{}|,\verb|\subsection{}| のコマンドを使います.
\verb|{}| の中にタイトルの文字列を入れてコンパイルすると章題目などが出力されます.
\textcolor{red}{一つの論文の中で同じ名前の章や節が存在することは望ましくありません.
名前の重複は避けましょう.
また,ある章の中に節が一つだけという状況も避けましょう(ある節の中に小節が一つだけという状況も同様です).}
節(小節)を設けるなら必ず複数設けて内容を分けましょう.
分けるつもりがないのであれば節(小節)を作らないようにしましょう.
また,この \verb|pdf| ファイルのソースコード中で \verb|\chapter{}| や \verb|\section{}| の次の行で \verb|\label{}| コマンドが使われているのがわかると思います.
これは \LaTeX の相互参照の機能を使うために各章・節にラベルをつけているのです.
詳細は第~\ref{ssec:ref} 節を参照してください.

\subsection{改行・改段落・空白}
\label{ssec:space}

Microsoft Word などでの文書作成に慣れた人は \LaTeX の改行や空白の扱いになかなか慣れないと思います.
まず改行について説明します.
\verb|tex| ファイル中で改行しても \verb|pdf| ファイルには反映されません.
したがって,文の途中で改行しても全く問題ありません.
次ページの枠内にある例では,【入力】で平家物語の冒頭が 1 文目から 4 文目までは 1 文ごとに改行されています.
しかし,【出力】では改行されずに前の文に続いて表示されています.
次に【入力】の 4 文目と 5 文目に注目しましょう.
間に空行が入っていますね.
この場合は【出力】で改段落しています.
\LaTeX の命令では空行が改段落を意味します.
\LaTeX では他にも改行の役割を担うコマンドが存在しますが,少しずつ違いがあります.
例えば \verb|\\| コマンドは「段落内の強制改行」なので改行後に冒頭一文字空きはありません.
文章中で段落を変える際に \verb|\\| で変えようとしている人をときどき見かけますが,これは適切な操作ではありません.
また,\verb|\par| コマンドで改段落している人もときどき見ますが,空白行を入れれば改段落できるので \verb|\par| コマンドを使うのは余計な手間ですよね.
逆に,改段落するつもりではない場所で空白行を入れてしまい,うっかり改段落してしまうというケースも見ます.
\LaTeX 初心者が引っ掛かりやすいポイントなので気をつけましょう.

\begin{tcolorbox}[enhanced, title=改行・改段落, drop fuzzy shadow]
【入力(\verb|tex| ファイルの中身)】
\begin{verbatim}
祇園精舍の鐘の声、諸行無常の響きあり。
娑羅双樹の花の色、盛者必衰の理をあらはす。
驕れる人も久しからず、ただ春の夜の夢のごとし。
猛き者もつひにはほろびぬ、ひとへに風の前の塵に同じ。

遠く異朝をとぶらへば、秦の趙高、漢の王莽、梁の朱异、唐の祿山、これらは皆舊主先皇の政にもしたがはず、樂しみをきはめ、諌めをも思ひ入れず、天下の亂れん事を悟らずして、民間の愁ふるところを知らざりしかば、久しからずして、亡じにし者どもなり。\\
近く本朝をうかがふに、承平の將門、天慶の純友、康和の義親、平治の信賴、これらはおごれる心もたけき事も、皆とりどりにこそありしかども、まぢかくは六波羅の入道、前太政大臣平朝臣清盛公と申しし人のありさま、傳へ承るこそ心もことばも及ばれね。
\end{verbatim}
\tcblower
【出力(\verb|pdf| ファイルでの見た目)】\\
 祇園精舎の鐘の声、諸行無常の響きあり。
沙羅双樹の花の色、盛者必衰の理をあらはす。
奢れる人も久からず、ただ春の夜の夢のごとし。
猛き者も遂にはほろびぬ、偏ひとへに風の前の塵におなじ。

 遠く異朝をとぶらへば、秦の趙高、漢の王莽、梁の朱异、唐の祿山、これらは皆舊主先皇の政にもしたがはず、樂しみをきはめ、諌めをも思ひ入れず、天下の亂れん事を悟らずして、民間の愁ふるところを知らざりしかば、久しからずして、亡じにし者どもなり。\\
近く本朝をうかがふに、承平の將門、天慶の純友、康和の義親、平治の信賴、これらはおごれる心もたけき事も、皆とりどりにこそありしかども、まぢかくは六波羅の入道、前太政大臣平朝臣清盛公と申しし人のありさま、傳へ承るこそ心もことばも及ばれね。
\end{tcolorbox}

次に \LaTeX での空白の取り扱いについて説明します.
ここの例では半角空白に関して説明します.
少々わかりにくいですが,【入力】では \verb|This| と \verb|is| の間に半角空白を一つ,\verb|is| と \verb|a| の間に半角空白を二つ,\verb|a| と \verb|pen.| の間に半角空白を三つ入れていますが,【出力】では無視されて一つ分の空白しか出てきません.

\begin{tcolorbox}[enhanced, title=空白の処理, drop fuzzy shadow]
【入力(\verb|tex| ファイルの中身)】
\begin{verbatim}
This is  a   pen.
\end{verbatim}
\tcblower
【出力(\verb|pdf| ファイルでの見た目)】\\
This is a pen.
\end{tcolorbox}

逆に空白を(自分の好きなサイズで)出力したい場合は \verb|\hspace{長さ}| や \verb|\vspace{長さ}| といったコマンドを使用します.

\subsection{相互参照}
\label{ssec:ref}


\section{\LaTeX での数式の書き方}
\label{sec:formula_in_LaTeX}


\begin{equation}
    \Re_\rmin = \frac{u_\rmin h}{\nu}, \quad \Re_\rmout = \frac{u_\rmout h}{\nu}, \quad \Re = \frac{u_0 h}{\nu}
\end{equation}

\begin{align}
    \frac{\partial u_i}{\partial t} + u_j\frac{\partial u_i}{\partial x_j} &= - \frac{1}{\rho}\frac{\partial p}{\partial x_j} + \nu\frac{\partial^2 u_i}{\partial x_j \partial x_j} \\
    \pdv{u_i}{t} + u_j\pdv{u_i}{x_j} &= - \frac{1}{\rho}\pdv{p}{x_j} + \nu\pdv{u_i}{x_j}{x_j}
\end{align}


