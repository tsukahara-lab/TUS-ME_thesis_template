\chapter{\BibTeX による参考文献一覧の出力}
\label{ch:bibtex}

第~\ref{ch:bibtex}~章では第~\ref{sec:bibcaution}~節で参考文献の一般的な記載方法を,第~\ref{sec:howtouse_bibtex}~節で \BibTeX の使用方法について説明します.

\section{参考文献の記載時の一般的な注意事項}
\label{sec:bibcaution}

第~\ref{sec:bibcaution}~節はこのテンプレートに限らない,一般的に学術論文等で参考文献を載せる際に注意すべき点をまとめています.
論文執筆の際には多くの文献を引用することになります.
読者に正しい情報を提供するのはもちろんのこと,先人たちの業績を認め,評価するという観点でも文献を引用する際は細心の注意を払いましょう.

\subsection{引用方式}
\label{ssec:citation_style}

参考文献の引用方法は Harvard 方式と Vancouver 方式に大別できます.
\begin{itemize}
    \item Harvard 方式
    \begin{itemize}
        \item 本文中での引用はいわゆる author-year 方式.「著者名」と「発行年」を記載する.
        \item 本文中での引用は苗字だけでの記載が多い.引用例:
        \begin{itemize}
            \item 著者 1 名:\cite{Reynolds:PhilTransRoySoc1883}
            \item 著者 2 名:\cite{Schmid:Springer2001}
            \item 著者 3 名以上:\cite{Berghout:JFM2020}
        \end{itemize}
        \item et al. はラテン語で「その他」を意味する et alii の略.\textit{Italic} 体で \textit{et al.} と書くことも多い.
        \item 論文末尾の文献リストは著者名のアルファベット順でソート.
    \end{itemize}
    \item Vancouver 方式
    \begin{itemize}
        \item 本文中での引用は番号.
        \item 本文中での引用例:~が明らかになっている $^{[1,2]}$.
        \item 論文末尾の文献リストは本文での登場順でソート.
    \end{itemize}
\end{itemize}
日本機械学会の論文執筆規定では Harvard 方式になっているため,この学位論文テンプレートも Harvard 方式を採用しています.

\subsection{文献リストの作り方}
\label{ssec:bib_list}

一般的な文献一覧の記載方法を説明します.
まず,Harvard 方式で文献を並べる際の一般的なソート規則は以下の通りです.
\begin{tcolorbox}[enhanced, title={文献ソート規則}, drop fuzzy shadow]
    \begin{enumerate}
        \item (Family, Givenの順で並べた際の)筆頭著者の氏名のアルファベット順.並べる際,日本人の氏名は漢字と仮名を用いた日本語表記で構わないが順序はアルファベット順とする.
        \item 筆頭著者が同一人物の場合,第二著者以降のアルファベット順で並べる.著者数が異なる場合は著者数が少ない方が先.これを最後の著者まで繰り返す.
        \item 著者が全員一致する文献があった場合は発行が早い順で並べる.
        \item 確認できる範囲で発行年月日が同じだった場合,タイトルのアルファベット順で並べ,(2022a) や (2022b) のように西暦の後に小文字でアルファベットを振る.
    \end{enumerate}
\end{tcolorbox}
次に,文献リストに記載する内容の一般的な注意事項を説明します.
このテンプレートのユーザーの皆さんは赤字の箇所にだけ注意しておけば,あとは \BibTeX が問題無く処理してくれるはずです.
\begin{tcolorbox}[enhanced, title={文献リスト作成の注意事項}, drop fuzzy shadow]
    \begin{itemize}
        \item 文献自体が日本語で書かれている場合は外国人が書いていても日本語文献とする.
        \item 文献自体が英語で書かれている場合は日本人が書いていても英語文献とする.
        \item 文献テンプレートによっては誌名を \textit{Italic} 体にしたり,巻数を \textbf{Boldface} 体にしたりすることがある.日本機械学会では全て Roman 体で統一する.
        \item \textcolor{red}{英語文献のタイトルは最初の単語の頭文字のみ大文字(固有名詞等は除く)\footnote{文頭の単語の頭文字のみ大文字で表記する方法を sentence case と言う.}.}
        \item \textcolor{red}{誌名・書名は冠詞や前置詞以外の各単語の頭文字を大文字にすることが多い\footnote{冠詞や前置詞以外の各単語の頭文字を大文字表記する方法を title case と言う.}.日本機械学会の規定では誌名・書名は省略せずに正式名称で記載する.}
        \item 「巻」は英語だと Vol.~xx,「号」は No.~xx や Issue.~xx と表記する.
        \item 発行年は西暦で表記する.括弧でくくるかどうかはテンプレート次第.
        \item ページ番号は単ページの場合は p.~xx とし,複数ページに亘る場合は pp.~xx--yy とする.
    \end{itemize}
\end{tcolorbox}

書籍の中には書誌情報が書かれている「奥付」に「刷」や「版」が記載されていることがあります.
書籍の発行年を調べる際にどの年号で記載するのかわからない人がいるのでここで説明します.
\begin{center}
    \begin{tabular}{ll}
        1970 年 9 月 12 日  & 初版発行 \\
        1980 年 4 月 6 日   & 第 2 版発行 \\
        1987 年 12 月 3 日  & 第 3 版発行 \\
        \textcolor{red}{1998} 年 1 月 25 日  & \textcolor{red}{第 4 版}発行 \\
        2002 年 4 月 8 日   & \textcolor{red}{第 4 版}第 2 刷発行 \\
        2004 年 10 月 18 日 & \textcolor{red}{第 4 版}第 3 刷発行 \\
        2008 年 7 月 14 日  & \textcolor{red}{第 4 版}第 4 刷発行
    \end{tabular}
\end{center}
例えば上記のような発行年一覧があったとしましょう.
書籍の内容が改訂されると「版」が変わります.
「刷」は印刷時期の違いなので内容は変わりません.
したがって,刷に関係なく参照した版の最初の発行年を記載してください.
上記の例で,参照した書籍が第 4 版だった場合,最初の発行年である 1998 年が記載すべき年号になります.

また,引用から引用するようないわゆる「孫引き」はしないでください.
オリジナルの文献を記載するようにしましょう.


\section{\BibTeX の使用方法}
\label{sec:howtouse_bibtex}

\LaTeX では文献リストを作る方法として \verb|thebibliography| 環境の中で \verb|\bibitem| コマンドを使用する方法があります.
しかし,この方法では文献リストを人間が全て手打ちで入力しなければいけません.
引用する文献が片手で数え切れるくらいの数であれば全て手打ちで文献リストを作ってもいいかもしれませんが,学術論文や学位論文になると人力で文献リストを作るのは時間の無駄ですしミスの元になります\footnote{学会の講演論文テンプレート等では,人力で文献リストを作る方法が採用されているものが結構あります.\texttt{\textbackslash bibitem} コマンド等を使って自力でリストを作成する方法は大抵の \LaTeX 入門書に記載があるのでそちらを参考にしてください.}.

そこで,\BibTeX を使用した参考文献リストの作成方法を説明します.
\BibTeX を使えばユーザーが作成した \verb|bib| ファイルを読み込んで自動で文献リストを作ってくれます\footnote{\BibTeX は現在でも広く使われていますが,最近は \BibTeX の後継として \texttt{biblatex} が徐々に普及してきています.この学位論文テンプレートでは後述の \texttt{jsme.bst} を使用しているため \BibTeX の内容に限定して記載しています.将来的には \texttt{biblatex} に置き換えたいと考えています.}.
一般的な文献の引用方法は第~\ref{sec:bibcaution}~節で説明しましたが,細かい規則は学会やジャーナルによって異なります.
それぞれの論文での引用ルールに則った出力を得るために必要なものが \BibTeX スタイルファイル,\verb|bst| ファイルです.
\BibTeX を走らせるときは \verb|bst| ファイルを読み込んで文献リストの出力方法を決めます.
有名な \verb|bst| ファイルとして \verb|jplain.bst| や \verb|jecon.bst| があります.

\subsection{\texttt{jsme.bst} について}
\label{ssec:jsme-bst}

東京理科大学創域理工学部機械航空宇宙工学科の卒業論文では,参考文献一覧および本文中での引用に関して一般社団法人日本機械学会の論文執筆テンプレートの書き方に沿って記載するよう決められています.
しかし,日本機械学会から公式な \BibTeX スタイルテンプレートは配布されていません.
そこで,塚原研究室所属の学生が日本機械学会の参考文献の書き方を再現した「非公式の」\BibTeX スタイルテンプレート\footnote{\texttt{JSME-bst}, \textless\url{https://github.com/Yuki-MATSUKAWA/JSME-bst}\textgreater}を開発し,GitHub で公開しているのでこれを使用します.
使用方法は一般的な \BibTeX と同様ですが,詳細な説明書(\href{https://github.com/Yuki-MATSUKAWA/JSME-bst/blob/main/JSME-template1.pdf}{\texttt{JSME-template1.pdf}})がリポジトリ内にあるので何か問題があった場合はそれを読むようにしましょう.
日本機械学会の規定通りの文献出力を得るには \verb|jsme.bst| を使用すれば大丈夫ですが,\verb|TUS-ME_thesis_template| リポジトリ内には最初から \verb|jsme.bst| が入っているのでこれを読んでいる皆さんが新たに \verb|jsme.bst| ファイルを \verb|JSME-bst| リポジトリから移してくる必要はありません.

\subsection{\texttt{bib} ファイルについて}
\label{ssec:bib-file}

\BibTeX は自動で文献リストを作ってくれるとは言ったものの,書誌情報は与えてあげないといけません.
\verb|bib| ファイルには自分が引用する書誌情報を記載します.
\verb|bib| ファイルの書き方は \verb|JSME-bst| 内の \href{https://github.com/Yuki-MATSUKAWA/JSME-bst/blob/main/JSME-template1.pdf}{\texttt{JSME-template1.pdf}} で詳細に書いてあるのでそちらをよく読んでください.
\BibTeX 初心者にとっても痒い所に手が届くように書かれています.
ただ,基本的な内容だけここにも書いておきます.

\verb|bib| ファイルに入力する書誌情報は次のような構造になっています.
\begin{tcolorbox}[enhanced, title={\texttt{bib} ファイル内の書誌情報の構造}, drop fuzzy shadow]
\begin{verbatim}
@エントリー名{参照キー,
    フィールド1 = {},
    フィールド2 = {},
    フィールド3 = {}
}
\end{verbatim}
\end{tcolorbox}
\noindent
だいたいの雑誌論文のウェブサイトでは \BibTeX 形式で書誌情報を出力できる機能があるのでそこから \verb|bib| ファイルをダウンロードします.
もちろん,ダウンロードした \verb|bib| ファイルを自分で書き換えることもできますし,自分で一から \verb|bib| ファイルを作成することも可能です.
文献を本文中で引用する際は \verb|\citet{Matsukawa:PoF2022}| のように書きます.
このときの \verb|Matsukawa:PoF2022| が参照キーです.
参照キーの書き方に特に規則は無く,半角カンマ以外の半角記号も使用可能です.
ただ,自分の中でマイルールを設けておくと引用する際に楽です.

\subsection{本文中での引用方法}
\label{ssec:cite}

このテンプレートでは \verb|natbib| パッケージを読み込んでいるため,本文中での文献引用方法が豊富に用意されています.
詳細は \verb|JSME-bst| の \href{https://github.com/Yuki-MATSUKAWA/JSME-bst/blob/main/JSME-template1.pdf}{\texttt{JSME-template1.pdf}} にも記載しているので,詳細はそちらを参照してください.
ただ,\verb|natbib| でサポートされているコマンドを全て使用する必要はなく,このテンプレートでの使用が想定されるのは以下のものです.
\textcolor{red}{下記コマンドのうち,最後の \texttt{\textbackslash citepe\{\}} のみ,このテンプレート用に作成したオリジナルのコマンドなので注意してください.
他の論文テンプレートでは使用できません.}

\begin{tcolorbox}[enhanced, title={本文中での文献引用コマンド(テンプレート用に一部改変・追加)}, drop fuzzy shadow]
    \begin{tabular}{ll}
        \textgt{コマンド}    & \textgt{出力} \\ \hline
        \verb|\citet{Matsukawa:PoF2022}|    & \citet{Matsukawa:PoF2022} \\
        \verb|\citet{松川:流力年会2022}|    & \citet{松川:流力年会2022} \\
        \verb|\citep{Matsukawa:PoF2022}|    & \citep{Matsukawa:PoF2022} \\
        \verb|\citep{松川:流力年会2022}|    & \citep{松川:流力年会2022} \\
        \verb|\citepe{Matsukawa:PoF2022}|   & \citepe{Matsukawa:PoF2022}
    \end{tabular}
\end{tcolorbox}

文献を通常の文章中に \citet{Matsukawa:PoF2022}, \citet{松川:流力年会2022} のように挿入する場合は \verb|\citet{}| で引用してください.
また,文末等に括弧に入れる形で\citep{Matsukawa:PoF2022,松川:流力年会2022}のように挿入する場合は \verb|\citep{}| で引用してください.
本来 \verb|\citep{}| コマンドで出力する括弧は半角括弧 () ですが,この学位論文テンプレートは日本語での執筆を想定しているので設定ファイル内で全角括弧()に変更しています.
この学位論文テンプレートではなく,海外のジャーナル論文を執筆する際は \verb|\citep{}| で半角括弧が出力されます.
ただし,日本語で論文を執筆している場合でも図表のキャプションは英語で書きます.
英文中に全角括弧が出現したら不自然なので,キャプション中は \verb|\citepe{}| というオリジナルのコマンドを使用してください.
また,図表のキャプション中で日本語文献を引用する場合は引用の出力を英語にしたいですね.
この場合は \verb|\citealias{}| コマンド や \verb|\citepalias{}| コマンドが便利です.
詳細は \verb|JSME-bst| の \href{https://github.com/Yuki-MATSUKAWA/JSME-bst/blob/main/JSME-template1.pdf}{\texttt{JSME-template1.pdf}} にも記載していますが,\verb|\defcitealias{}| コマンドを使い,読み込む文献と出力結果を自分で定義することで任意の形での出力が可能です.



