\chapter{\LaTeX の基本}
\label{ch:basic}


\section{\LaTeX での文章の書き方}
\label{sec:sentence_in_LaTeX}

\subsection{章・節・小節}
\label{ssec:ch_sec_ssec}

この学位論文テンプレートは \verb|report| と呼ばれる文書クラスを使用しているため,章(chapter),節(section),小節(subsection)に分けて文章を書けます.
例えば今読んでいるこの文章は第~\ref{ch:basic}~章の第~\ref{ssec:ch_sec_ssec}~節に位置しています.
それぞれの章や節のタイトルをつけるには \verb|\chapter{}|,\verb|\section{}|,\verb|\subsection{}| のコマンドを使います.
\verb|{}| の中にタイトルの文字列を入れてタイプセットすると章題目などが出力されます.
\textcolor{red}{ある章の中に節が一つだけという状況は避けましょう(ある節の中に小節が一つだけという状況も同様です).}
節(小節)を設けるなら必ず複数設けて内容を分けましょう.
分けるつもりがないのであれば節(小節)を作らないようにしましょう.
また,この \verb|pdf| ファイルのソースコード中で \verb|\chapter{}| や \verb|\section{}| の次の行で \verb|\label{}| コマンドが使われているのがわかると思います.
これは \LaTeX の相互参照の機能を使うために各章・節にラベルをつけているのです.
詳細は第~\ref{ssec:ref}~節を参照してください.

\subsection{改行・改段落・空白}
\label{ssec:space}

Microsoft Word などでの文書作成に慣れた人は \LaTeX の改行や空白の扱いになかなか慣れないと思います.
まず改行について説明します.
\verb|tex| ファイル中で改行しても \verb|pdf| ファイルには反映されません.
したがって,文の途中で改行しても全く問題ありません.
次ページの枠内にある例では,【入力】で平家物語の冒頭が 1 文目から 4 文目までは 1 文ごとに改行されています.
しかし,【出力】では改行されずに前の文に続いて表示されています.
次に【入力】の 4 文目と 5 文目に注目しましょう.
間に空行が入っていますね.
この場合は【出力】で改段落しています.
\LaTeX の命令では空行が改段落を意味します.
\LaTeX では他にも改行の役割を担うコマンドが存在しますが,少しずつ違いがあります.
例えば \verb|\\| コマンドは「段落内の強制改行」なので改行後に冒頭一文字空きはありません.
文章中で段落を変える際に \verb|\\| で変えようとしている人をときどき見かけますが,これは適切な操作ではありません.
また,\verb|\par| コマンドで改段落している人もときどき見ますが,空白行を入れれば改段落できるので \verb|\par| コマンドを使うのは余計な手間ですよね.
逆に,改段落するつもりではない場所で空白行を入れてしまい,うっかり改段落してしまうというケースも見ます.
\LaTeX 初心者が引っ掛かりやすいポイントなので気をつけましょう.

\begin{tcolorbox}[title=改行・改段落]
【入力(\verb|tex| ファイルの中身)】
\begin{verbatim}
祇園精舍の鐘の声、諸行無常の響きあり。
娑羅双樹の花の色、盛者必衰の理をあらはす。
驕れる人も久しからず、ただ春の夜の夢のごとし。
猛き者もつひにはほろびぬ、ひとへに風の前の塵に同じ。

遠く異朝をとぶらへば、秦の趙高、漢の王莽、梁の朱异、唐の祿山、これらは皆舊主先皇の政にもしたがはず、樂しみをきはめ、諌めをも思ひ入れず、天下の亂れん事を悟らずして、民間の愁ふるところを知らざりしかば、久しからずして、亡じにし者どもなり。\\
近く本朝をうかがふに、承平の將門、天慶の純友、康和の義親、平治の信賴、これらはおごれる心もたけき事も、皆とりどりにこそありしかども、まぢかくは六波羅の入道、前太政大臣平朝臣清盛公と申しし人のありさま、傳へ承るこそ心もことばも及ばれね。
\end{verbatim}
\tcblower
【出力(\verb|pdf| ファイルでの見た目)】\\
 祇園精舎の鐘の声、諸行無常の響きあり。
沙羅双樹の花の色、盛者必衰の理をあらはす。
奢れる人も久からず、ただ春の夜の夢のごとし。
猛き者も遂にはほろびぬ、偏ひとへに風の前の塵におなじ。

 遠く異朝をとぶらへば、秦の趙高、漢の王莽、梁の朱异、唐の祿山、これらは皆舊主先皇の政にもしたがはず、樂しみをきはめ、諌めをも思ひ入れず、天下の亂れん事を悟らずして、民間の愁ふるところを知らざりしかば、久しからずして、亡じにし者どもなり。\\
近く本朝をうかがふに、承平の將門、天慶の純友、康和の義親、平治の信賴、これらはおごれる心もたけき事も、皆とりどりにこそありしかども、まぢかくは六波羅の入道、前太政大臣平朝臣清盛公と申しし人のありさま、傳へ承るこそ心もことばも及ばれね。
\end{tcolorbox}

次に \LaTeX での空白の取り扱いについて説明します.
ここの例では半角空白に関して説明します.
少々わかりにくいですが,【入力】では \verb|This| と \verb|is| の間に半角空白を一つ,\verb|is| と \verb|a| の間に半角空白を二つ,\verb|a| と \verb|pen.| の間に半角空白を三つ入れていますが,【出力】では無視されて一つ分の空白しか出てきません.

\begin{tcolorbox}[title={空白の処理}]
【入力(\verb|tex| ファイルの中身)】
\begin{verbatim}
This is  a   pen.
\end{verbatim}
\tcblower
【出力(\verb|pdf| ファイルでの見た目)】\\
This is a pen.
\end{tcolorbox}

逆に空白を(自分の好きなサイズで)出力したい場合は \verb|\hspace{長さ}| や \verb|\vspace{長さ}| といったコマンドを使用します.

\subsection{相互参照}
\label{ssec:ref}

\LaTeX で文書を書くメリットの一つに相互参照の機能が充実していることが挙げられます.
相互参照は「第~\ref{ssec:ref}~節を参照されたい.」や「式~\eqref{eq:NSr} を代入すると~」のような文脈において,対応する章・節・式・図・表などの番号を文書中から探し出す機能です.
また,ハイパーリンクを有効にしておくことで,\verb|pdf| ファイルに適切なリンクが埋め込まれ,クリックで該当箇所に飛ぶことができ大変便利です.
この学位論文テンプレートでもハイパーリンクを有効化しており,青字の文字列をクリックすると参照先に飛べます.
図表や章題目などを \verb|\label{}| コマンドでラベリングし,参照する文章中で \verb|\ref{}| コマンドを使って呼び出すのが基本的な相互参照の形です.
また,数式を参照する際は \verb|\eqref{}| コマンドの使用が一般的です.
図表や式をラベリングしておくことで,式番号や図番号が変わったとしてもその変化に合わせて \verb|pdf| ファイルの出力も変えられます.
また,論文を書き進める途中で章や節の順番が丸ごと入れ替わるといった事態が起きても,相互参照の機能を使っていれば全く問題ありません.
ただし,同じ名前のラベルは使用できないため,名前の重複には気をつけましょう.
存在しないラベル名を参照してもエラーとなり,\verb|pdf| ファイル中での出力は ?? となります.


\section{\LaTeX での数式の書き方}
\label{sec:formula_in_LaTeX}

\subsection{基本的な数式の記法}
\label{ssec:basic_command}

最も基本的な別行立ての数式は \verb|equation| 環境で記述します.
\begin{equation}
    \frac{\partial u_r}{\partial t} + (\mathbf{u}\cdot\nabla)u_r - \frac{u_\theta^2}{r} = -\frac{1}{\rho}\frac{\partial p}{\partial r} + \nu\left(\nabla^2 u_r - \frac{u_r}{r^2} - \frac{2}{r^2}\frac{\partial u_\theta}{\partial \theta}\right)
    \label{eq:NSr}
\end{equation}
式~\eqref{eq:NSr} の数式は \LaTeX でサポートされている最も標準的なコマンドで記述しています.
上付き添え字はキャレット \verb|^|,下付き添え字はアンダースコア \verb|_| で表現します.
したがって,$u_\theta^2$ は \verb|u_\theta^2| と書きます.
ここで注意点として,添え字が $u_\theta^2$ のように一文字であれば問題無いのですが,$R_{ij}$ のように二文字以上の場合は \verb|R_{ij}| のように括弧 \verb|{}| で囲んでください.
分数は \verb|\frac{分子}{分母}| で書き,常微分・偏微分の場合も同様です.
表~\ref{table:greek} のように $\rho$ や $\theta$ のといったギリシャ文字も出力できるほか,$\nabla$ や $\sin$,$\log$ のような数学で使う関数の類もコマンドが存在しています.
$\sin$ や $\log$ は通常アップライト体(立体,Roman 体)で書きます.
$sin x$ などと書くことのないように気をつけましょう.
また,数式中で \LaTeX コマンドを使わずに変換で出した全角のθを入れている人をときどき見かけるので気をつけましょう.
式~\eqref{eq:NSr} の左辺と右辺で丸括弧の大きさが異なることにも注目してください.
左辺は特に何もしていませんが,右辺は分数がある分,縦方向に括弧の長さが必要です.
括弧の大きさを自動で調整するコマンドとして,\verb|\left(|,\verb|\right)| コマンドがあります.
左辺第二項では速度 $\mathbf{u}$ がベクトルであるため,\textbf{Bold} 体になっています.
これは \verb|\mathbf{u}| とすることで出力できます.
このテンプレートでは \verb|unicode-math| を使用しているので \verb|\symbf{u}| というコマンドもサポートされています.
ベクトルをボールドイタリック体 $\symbfit{u}$ にしたい場合は \verb|$\symbfit{u}$| というコマンドを使用してください(第~\ref{ssec:typeface_math}~節を参照).
このテンプレートでは \verb|bm| パッケージを使用していません.

式~\eqref{eq:NSr} は別行立ての数式でしたが,$E = mc^2$ のように文章中に組み込む数式(インライン数式)を出力する場合は \verb|$$| で数式を囲って \verb|$E = mc^2$| と書きます.

複数の数式を並べる場合は \verb|align| 環境を使いましょう.
\begin{align}
    a^2 &= b^2 + c^2 - 2bc\cos A \label{eq:cosA} \\
    b^2 &= c^2 + a^2 - 2ca\cos B \label{eq:cosB} \\
    c^2 &= a^2 + b^2 - 2ab\cos C \label{eq:cosC}
\end{align}
\verb|align| 環境では \verb|&| の位置で揃えるように制御しています.
式~\eqref{eq:cosA}--\eqref{eq:cosC} は $=$ の前に \verb|&| を置いているので,3 式の位置は $=$ の直前で揃っていることになります.
\verb|align| 環境を使えば
\begin{align}
    \sin 2\alpha &= \sin(\alpha + \alpha) \notag \\
    &= \sin\alpha\cos\alpha + \cos\alpha\sin\alpha \notag \\
    &= 2\sin\alpha\cos\alpha
    \label{eq:double-sin}
\end{align}
のように途中式も入れられます.
式番号を振らなくていい行は \verb|\notag| コマンドを使用します.
\begin{align}
    \varepsilon\left.\ab(\pdv[order=2]{\tilde{\phi}_1}{t} + g\pdv{\tilde{\phi}_1}{z})\right|_{z=0} &+ \varepsilon^2\left[\pdv[order=2]{\tilde{\phi}_2}{t} + g\pdv{\tilde{\phi}_2}{z} + \tilde{\eta}_1\pdv{}{z}\ab(\pdv[order=2]{\tilde{\phi}_1}{t} + g\pdv{\tilde{\phi}_1}{z})\right. \notag \\
    &\left.\left.+ \pdv{}{t}\ab\{\ab(\pdv{\tilde{\phi}_1}{x}) + \ab(\pdv{\tilde{\phi}_1}{z})^2\}\right]\right|_{z=0} = 0
    \label{eq:multiple-lines}
\end{align}
式~\eqref{eq:multiple-lines} は式の途中で改行を挟む場合の処理方法を示しています.
ただしソースコードは一部,第~\ref{ssec:physics2}~節で紹介する \verb|physics2| パッケージを使用しているので注意してください.
式~\eqref{eq:multiple-lines} で注目すべき箇所は大括弧 $[\quad]$ の途中で改行を挟んでいる点,代入記法として $|_{z=0}$ を使用している点の二箇所です.
一つの行内で \verb|\left| コマンドに対応する \verb|\right| コマンドが無いとエラーとなります.
大括弧 $[\quad]$ 内で改行する際は一行目で \verb|\left[| に対して \verb|\right.| を使用し,二行目で \verb|\right]| に対して \verb|\left.| を使用することで対処しています.
\verb|\left.| と \verb|\right.| は何も出力されません.
\verb|\left| または \verb|\right| に対応するものを用意するために使用します.
同様に,代入記法としての $|_{z=0}$ は \verb"\right|_{z=0}" に対して \verb|\left.| を使用しています.

場合分けのある数式は \verb|cases| 環境が便利です.
式~\eqref{eq:Kronecker} は Kronecker のデルタです.
\begin{equation}
    \delta_{ij} = 
    \begin{cases}
        1, & i = j \\
        0, & i \neq j
    \end{cases}
    \label{eq:Kronecker}
\end{equation}

\begin{table}[tp]
    \centering
    \caption{ギリシャ文字の出力方法.}
    \label{table:greek}
    \begin{tabular}{c|c|c|c|c} \hline\hline
        英語名 & 大文字出力 & 大文字入力 & 小文字出力 & 小文字入力 \\ \hline
        alpha & $A$ & \verb|A| & $\alpha$ & \verb|\alpha| \\ \hline
        beta & $B$ & \verb|B| & $\beta$ & \verb|\beta| \\ \hline
        gamma & $\Gamma$, $\varGamma$ & \verb|\Gamma|, \verb|\varGamma| & $\gamma$ & \verb|\gamma| \\ \hline
        delta & $\Delta$, $\varDelta$ & \verb|\Delta|, \verb|\varDelta| & $\delta$ & \verb|\delta| \\ \hline
        epsilon & $E$ & \verb|E| & $\epsilon$, $\varepsilon$ & \verb|\epsilon|, \verb|\varepsilon| \\ \hline
        zeta & $Z$ & \verb|Z| & $\zeta$ & \verb|\zeta| \\ \hline
        eta & $\Eta$ & \verb|\Eta| & $\eta$ & \verb|\eta| \\ \hline
        theta & $\Theta$, $\varTheta$ & \verb|\Theta|, \verb|\varTheta| & $\theta$, $\vartheta$ & \verb|\theta|, \verb|\vartheta| \\ \hline
        iota & $I$ & \verb|I| & $\iota$ & \verb|\iota| \\ \hline
        kappa & $K$ & \verb|K| & $\kappa$, $\varkappa$ & \verb|\kappa|, \verb|\varkappa| \\ \hline
        lambda & $\Lambda$, $\varLambda$ & \verb|\Lambda|, \verb|\varLambda| & $\lambda$ & \verb|\lambda| \\ \hline
        mu & $M$ & \verb|M| & $\mu$ & \verb|\mu| \\ \hline
        nu & $N$ & \verb|N| & $\nu$ & \verb|\nu| \\ \hline
        xi & $\Xi$, $\varXi$ & \verb|\Xi|, \verb|\varXi| & $\xi$ & \verb|\xi| \\ \hline
        omicron & $O$ & \verb|O| & $o$ & \verb|o| \\ \hline
        pi & $\Pi$, $\varPi$ & \verb|\Pi|, \verb|\varPi| & $\pi$, $\varpi$ & \verb|\pi|, \verb|\varpi| \\ \hline
        rho & $P$ & \verb|P| & $\rho$, $\varrho$ & \verb|\rho|, \verb|\varrho| \\ \hline
        sigma & $\Sigma$, $\varSigma$ & \verb|\Sigma|, \verb|\varSigma| & $\sigma$, $\varsigma$ & \verb|\sigma|, \verb|\varsigma| \\ \hline
        tau & $T$ & \verb|T| & $\tau$ & \verb|\tau| \\ \hline
        upsilon & $\Upsilon$, $\varUpsilon$ & \verb|\Upsilon|, \verb|\varUpsilon| & $\upsilon$ & \verb|\upsilon| \\ \hline
        phi & $\Phi$, $\varPhi$ & \verb|\Phi|, \verb|\varPhi| & $\phi$, $\varphi$ & \verb|\phi|, \verb|\varphi| \\ \hline
        chi & $X$ & \verb|X| & $\chi$ & \verb|\chi| \\ \hline
        psi & $\Psi$, $\varPsi$ & \verb|\Psi|, \verb|\varPsi| & $\psi$ & \verb|\psi| \\ \hline
        omega & $\Omega$, $\varOmega$ & \verb|\Omega|, \verb|\varOmega| & $\omega$ & \verb|\omega| \\ \hline
    \end{tabular}
\end{table}

\subsection{\texttt{physics2} パッケージ}
\label{ssec:physics2}

式~\eqref{eq:NSr} は基本的なコマンドを紹介するためにわざと面倒な書き方をしました.
しかし,複雑な数式は書くのも大変ですしミスの元になります.
また,数式のコードが長くなるとエラーの原因を探すのも難しくなります.
できるだけ数式を簡単に書くためのツールとして \verb|physics| パッケージが開発され,現在まで広く使われています.
ただ,\verb|physics| パッケージはいくつかの問題点を抱えており,最近は \verb|physics| パッケージの代替となる \verb|physics2| パッケージが開発されています.
\verb|physics2| パッケージは現在も開発途上のパッケージであるため,使い方は公式ドキュメントをよく読んでください.
\verb|physics2| パッケージでは常微分・偏微分用のコマンドはサポートされていないので,このテンプレートでは \verb|fixdif| パッケージと \verb|derivative| パッケージを併せて使用しています.
\begin{equation}
    \pdv{u_r}{t} + (\mathbf{u}\cdot\nabla)u_r - \frac{u_\theta^2}{r} = -\frac{1}{\rho}\pdv{p}{r} + \nu\ab(\nabla^2 u_r - \frac{u_r}{r^2} - \frac{2}{r^2}\pdv{u_\theta}{\theta})
    \label{eq:NSr2}
\end{equation}
式~\eqref{eq:NSr2} は式~\eqref{eq:NSr} を \verb|physics2| パッケージなどで書き換えたものです.
偏微分と大きさ調整が必要な括弧の記述が幾分か楽になりましたね.
また,微積分で使用する $\d$ は物理量の $d$ と区別するために直立体で表記することが推奨されています.
\verb|fixdif| パッケージのコマンドで $\d x$ と出力するには \verb|\d x| と入力します.
また,常微分の記法としては \verb|derivative| パッケージの \verb|\odv| コマンドを使用します.

\subsection{\texttt{siunitx} パッケージ}
\label{ssec:siunitx}

数式中の物理量は \textit{Italic} 体で表記しますが,単位は直立体で表記するのが一般的です.
また,数値と単位の間には空白を設けるのが一般的な書き方です\footnote{例外的に空白を設けなくてもいい単位として,角度を表す $\si{\degree}$ があります.$\ang{45}$ のように数値と単位を詰めて書くことが許容されています.ただし,温度を表す $\si{\degreeCelsius}$ は空白が必要です($\SI{45}{\degreeCelsius}$).}.
これらの要求を満たして簡単に単位を書けるのが \verb|siunitx| パッケージです.
\verb|siunitx| パッケージでは大きく分けてテキストモードとマクロモードの二種類の書き方があります.
テキストモードでは \verb|\si{W/m^2 K}| の数式のように書けるのに対し,マクロモードでは \verb|\si{\watt/\meter^2\kelvin}| のようにそれぞれの単位で用意されているコマンドで出力します.
出力結果はどちらも $\si{W/m^2 K}$ です.
$\si{\degreeCelsius}$ のようにマクロモードでしか使用できない単位もあります.
\verb|siunitx| パッケージは非常に便利なので以下のコマンドを積極的に使用しましょう.
単位のみの出力は \verb|\si{}| コマンド,数値と単位を併せての出力は \verb|\SI{}| コマンドを使用します.
\verb|\SI{}| コマンドを使用すると,数値と単位の間に適切な長さの空白を自動で入れてくれます.

\begin{tcolorbox}[title={\texttt{siunitx} パッケージ}]
    \begin{tabular}{lll}
        【テキストモードとマクロモードと混在】\\
        \textgt{コマンド}  & \textgt{出力} \\ \hline
        \verb|\si{W/m^2 K}|   & $\si{W/m^2 K}$ \\
        \verb|\si[per-mode=symbol]{\watt\per{\square\meter\kelvin}}|  & $\si[per-mode=symbol]{\watt\per{\square\meter\kelvin}}$ \\
        \verb|\si{\watt/\meter^2\kelvin}|  & $\si{\watt/\meter^2\kelvin}$
    \end{tabular}
    \tcblower
    \begin{tabular}{ll}
        【さまざまな単位の出力】\\
        \textgt{コマンド}  & \textgt{出力} \\ \hline
        \verb|\SI{45}{W/m^2 K}|   & $\SI{45}{W/m^2 K}$ \\
        \verb|\SI{45}{\degreeCelsius}|  & $\SI{45}{\degreeCelsius}$ \\
        \verb|\SI{45}{\micro\meter}|    & $\SI{45}{\micro\meter}$ \\
        \verb|\SI{45}{\um}|    & $\SI{45}{\um}$ \\
        \verb|\SI{45}{mL}|    & $\SI{45}{mL}$ \\
    \end{tabular}
\end{tcolorbox}

最後に示したミリリットル $\si{mL}$ には気をつけてください.
リットル $\si{L}$ を昔は $\ell$ と表記したこともありましたが,「単位は直立体」という原則に合わないのでやめましょう.
また,小文字の $\si{l}$ だと数字の $1$ と紛らわしいので,人名由来の単位ではありませんがリットルは大文字 $\si{L}$ で書くようにしましょう.

\begin{tcolorbox}[title={第~\ref{ch:basic}~章の参考文献}, colback=yellow!5!white, colframe=yellow!75!black, coltitle=black]
    \begin{itemize}
        \item 奥村晴彦, 黒木裕介,[改訂第 9 版]\LaTeX 美文書作成入門, 技術評論社 (2023), pp.~25--40, 73--115.
        \item \href{https://qiita.com/Yarakashi_Kikohshi/items/131e2324f401c3effb84}{Qiita: 脱 physics パッケージして physics2 パッケージを使おう}
        \item \href{https://qiita.com/gawara-t/items/57834e06f7fd95c18d26}{Qitta:【LaTeX】physicsパッケージとphysics2パッケージ(+α)との対応一覧}
        \item \href{http://www.yamamo10.jp/yamamoto/comp/latex/make_doc/unit/index.php}{Yamamot's Laboratory: LaTeX 数値・単位(siunitx)}
    \end{itemize}
\end{tcolorbox}


