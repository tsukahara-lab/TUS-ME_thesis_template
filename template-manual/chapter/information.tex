\chapter{さらに詳しい情報が欲しい人は}
\label{ch:information}

\section{論文の書き方に関する情報}
\label{sec:thesisinfo}

論文の書き方に関して参考になる文献を挙げますが,他にもいい書籍はたくさんあります.

\begin{itemize}
    \item 木下是雄, 理科系の作文技術, 中公新書 (1981).
    \item 本多勝一, <新版>日本語の作文技術, 朝日新聞出版 (2015).
\end{itemize}

\section{\TeX/\LaTeX に関する情報}
\label{sec:latexinfo}

タイプセットの最中にエラーが発生して PDF ファイルを作成できなかったときは,まずは \verb|log| ファイルのエラーメッセージを確認しましょう.
デフォルトでは \verb|latex.out/main.log| というファイル名で出力されます.
エラーメッセージの内容を読んでも原因がわからない場合はインターネットで検索するか書籍を参照するようにしましょう.
もっと根本的に,\LaTeX でサポートされている機能の詳細を調べたい場合は TeX Live 公式ドキュメントを読むようにしましょう.
例えば \verb|siunitx| パッケージについて調べたい場合は
\begin{tcolorbox}
\begin{verbatim}
$ texdoc siunitx
\end{verbatim}
\end{tcolorbox}
とターミナルに入力すると \verb|siunitx| パッケージの公式ドキュメントが出てきます.
また,このテンプレートマニュアル自体も初心者向けの情報はそれなりに載っています.
せっかく作ったのだからよく読んでください.
問題が発生したらすぐに先生や先輩に聞くのではなく,まずはこれらの手段を使って自分で原因を究明するようにしましょう.

\subsection{書籍}
\label{ssec:book}

\TeX/\LaTeX に関する書籍は多く存在しますが,古い情報が載っているものもあるのでできるだけ新しい書籍を読むようにしましょう.
また,初心者が難しい書籍を読んでも何も理解できないのでここでは「初心者がこの学位論文テンプレートマニュアルの内容を超えた \TeX/\LaTeX に関する情報を入手したい場合に読むべき書籍」を列挙します.
基本的には奥村先生の『美文書作成入門』が網羅的に書かれているので研究室にこれが一冊あれば問題ないでしょう.
吉永先生の『\LaTeXe 辞典』は少々難易度が上がりますが,\LaTeX で実行したい少し高度な編集に役立ちます.

\begin{itemize}
    \item 奥村晴彦, 黒木裕介,[改訂第 9 版]\LaTeX 美文書作成入門, 技術評論社 (2023).
    \item 土屋勝, \LaTeX はじめの一歩 Windows 11/10 対応, カットシステム (2022).
    \item 吉永徹美, \LaTeXe 辞典 増補改訂版, 翔泳社 (2018).
\end{itemize}

\subsection{インターネット上の情報}
\label{sec:internet}

\TeX/\LaTeX に関する情報はインターネットに大量に転がっているので大抵のことは検索すれば解決します.
ただ,インターネット上の \TeX/\LaTeX の情報は古いもの・現在非推奨のものが大量にあるので注意が必要です.
また,個人ブログのようなものに書かれている情報は環境依存の場合もあるので気をつけましょう.
インターネット上の情報である程度信頼性が高いものを列挙します.

\begin{itemize}
    \item \href{https://texwiki.texjp.org/}{\TeX{} Wiki}: \TeX/\LaTeX に関するあれこれが書かれている日本語のウェブページ.検索したらここに行きつくことが多い.
    \item \href{https://ctan.org/}{CTAN}: \TeX/\LaTeX に関連する情報やファイルが集まっている.英語.
\end{itemize}


