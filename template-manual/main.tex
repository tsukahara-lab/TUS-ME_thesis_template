%%%%%%%%%%%%%%%%%%%%%%%%%%%%%%%%%%%%%%%%%%%%%%%%%%%%%%%%%%%%%%%%%%%%%%%
%%%
%%%                東京理科大学 創域理工学部 機械航空宇宙工学科
%%%                   【非公式】学位論文 LaTeX テンプレート
%%%
%%%       <https://github.com/tsukahara-lab/TUS-ME_thesis_template>
%%%
%%%                                  v1.0.0 Yuki MATSUKAWA 27 Dec. 2023
%%%                                  v2.0.0 Yuki MATSUKAWA 15 Nov. 2024
%%%                                  v2.1.1 Yuki MATSUKAWA 26 Aug. 2025
%%%
%%%%%%%%%%%%%%%%%%%%%%%%%%%%%%%%%%%%%%%%%%%%%%%%%%%%%%%%%%%%%%%%%%%%%%%

%%% 文書クラスの設定 %%%
\documentclass[
    paper=a4paper,      % A4 用紙サイズ
    report,             % report 相当の文書クラス
    fleqn,              % 数式を左寄せ
    fontsize=12pt,      % 欧文サイズ 12 pt
    jafontsize=12pt,    % 和文サイズ 12 pt
    head_space=33mm,    % 天の余白(柱とノンブルがあるので 20 mm よりも広い)
    foot_space=30mm,    % 地の余白(ノンブルが下の場合があるので 15 mm よりも広い)
    gutter=25mm,        % のどの余白
    fore-edge=10mm,     % 小口の余白
    % draft               % 画像を省略(タイプセットを高速化.提出時はコメントアウト)
    ]{jlreq}            % jlreq クラスを使用

%%% 学位論文設定ファイル %%%
\usepackage{settings}

%%% 行番号の表示 %%%
% 添削時には行番号を付けるとわかりやすい
% 提出時にはコメントアウトする
% \linenumbers

%%% ここから上を「プリアンブル」と言います.パッケージや独自の設定,マクロはプリアンブルや settings.styに書いてください.
%%% ここから下が論文の本体です.
\begin{document}

%%%%%%%%%%%%%%%%
%%%%% 表紙 %%%%%
%%%%%%%%%%%%%%%%

% 卒業・修了「年度」を入力
\thesis{20**年度卒業論文}   % 卒業論文はこれ
% \thesis{20**年度修士論文}   % 修士論文はこれ

% 学位論文題目
% ここには学位論文のタイトルを入れます.一文字でも間違えたら受理されません.
% タイトルが長くて改行するときは \\ を入れる.
\title{【非公式】機械航空宇宙工学科 学位論文テンプレート\\ \――\LaTeX で論文を書く際に必要な最低限の情報\――}

% 卒業・修了「年」を入力
\date{20**年2月}

% 卒業論文の場合はこれ
% 大学名,学部名,学科名の間にスペースは不要
\affiliation{東京理科大学創域理工学部機械航空宇宙工学科}

% 修士論文の場合はこれ
% 大学名,研究科名,専攻名の間にスペースは不要
% \affiliation{東京理科大学大学院創域理工学研究科機械航空宇宙工学専攻}

% 研究室名を入力
\laboratory{〇〇研究室}

% 著者情報
\author{%
% 学籍番号を全角 7 桁で入力
75*****
\hskip2\zw% 学籍番号と氏名の間のスペース,消さない
% 姓と名の間は全角 1 文字スペース
姓姓 名名
} % 消さない

% 表紙の出力
\makecover

%%% 目次 %%%
\tableofcontents

%%% 記号表 %%%
\input{chapter/signary.tex}

%%%%%%%%%%%%%%%%
%%%%% 本文 %%%%%
%%%%%%%%%%%%%%%%
\clearpage
\pagestyle{normal}
\setcounter{page}{0}
\pagenumbering{arabic}
% 上のコマンドは消さないで.
% 本文はこれ以降に記載する.

% LaTeX ソースは一つの tex ファイルに書くのではなく,章ごとの tex ファイルに分割して書きましょう.
% 分割したファイルを読み込むときは \input{xxx} または \include{xxx} を使います.

%%% はじめに %%%
\input{chapter/introduction.tex}

%%% 環境構築・操作方法 %%%
\input{chapter/howtouse.tex}

%%% LaTeX の基本 %%%
\chapter{\LaTeX の基本}
\label{ch:basic}


\section{\LaTeX での文章の書き方}
\label{sec:sentence_in_LaTeX}

\subsection{章・節・小節}
\label{ssec:ch_sec_ssec}

この学位論文テンプレートは \verb|report| と呼ばれる文書クラスを使用しているため,章(chapter),節(section),小節(subsection)に分けて文章を書けます.
例えば今読んでいるこの文章は第~\ref{ch:basic}~章の第~\ref{ssec:ch_sec_ssec}~節に位置しています.
それぞれの章や節のタイトルをつけるには \verb|\chapter{}|,\verb|\section{}|,\verb|\subsection{}| のコマンドを使います.
\verb|{}| の中にタイトルの文字列を入れてタイプセットすると章題目などが出力されます.
\textcolor{red}{ある章の中に節が一つだけという状況は避けましょう(ある節の中に小節が一つだけという状況も同様です).}
節(小節)を設けるなら必ず複数設けて内容を分けましょう.
分けるつもりがないのであれば節(小節)を作らないようにしましょう.
また,この \verb|pdf| ファイルのソースコード中で \verb|\chapter{}| や \verb|\section{}| の次の行で \verb|\label{}| コマンドが使われているのがわかると思います.
これは \LaTeX の相互参照の機能を使うために各章・節にラベルをつけているのです.
詳細は第~\ref{ssec:ref}~節を参照してください.

\subsection{改行・改段落・空白}
\label{ssec:space}

Microsoft Word などでの文書作成に慣れた人は \LaTeX の改行や空白の扱いになかなか慣れないと思います.
まず改行について説明します.
\verb|tex| ファイル中で改行しても \verb|pdf| ファイルには反映されません.
したがって,文の途中で改行しても全く問題ありません.
次ページの枠内にある例では,【入力】で平家物語の冒頭が 1 文目から 4 文目までは 1 文ごとに改行されています.
しかし,【出力】では改行されずに前の文に続いて表示されています.
次に【入力】の 4 文目と 5 文目に注目しましょう.
間に空行が入っていますね.
この場合は【出力】で改段落しています.
\LaTeX の命令では空行が改段落を意味します.
\LaTeX では他にも改行の役割を担うコマンドが存在しますが,少しずつ違いがあります.
例えば \verb|\\| コマンドは「段落内の強制改行」なので改行後に冒頭一文字空きはありません.
文章中で段落を変える際に \verb|\\| で変えようとしている人をときどき見かけますが,これは適切な操作ではありません.
また,\verb|\par| コマンドで改段落している人もときどき見ますが,空白行を入れれば改段落できるので \verb|\par| コマンドを使うのは余計な手間ですよね.
逆に,改段落するつもりではない場所で空白行を入れてしまい,うっかり改段落してしまうというケースも見ます.
\LaTeX 初心者が引っ掛かりやすいポイントなので気をつけましょう.

\begin{tcolorbox}[title=改行・改段落]
【入力(\verb|tex| ファイルの中身)】
\begin{verbatim}
祇園精舍の鐘の声、諸行無常の響きあり。
娑羅双樹の花の色、盛者必衰の理をあらはす。
驕れる人も久しからず、ただ春の夜の夢のごとし。
猛き者もつひにはほろびぬ、ひとへに風の前の塵に同じ。

遠く異朝をとぶらへば、秦の趙高、漢の王莽、梁の朱异、唐の祿山、これらは皆舊主先皇の政にもしたがはず、樂しみをきはめ、諌めをも思ひ入れず、天下の亂れん事を悟らずして、民間の愁ふるところを知らざりしかば、久しからずして、亡じにし者どもなり。\\
近く本朝をうかがふに、承平の將門、天慶の純友、康和の義親、平治の信賴、これらはおごれる心もたけき事も、皆とりどりにこそありしかども、まぢかくは六波羅の入道、前太政大臣平朝臣清盛公と申しし人のありさま、傳へ承るこそ心もことばも及ばれね。
\end{verbatim}
\tcblower
【出力(\verb|pdf| ファイルでの見た目)】\\
 祇園精舎の鐘の声、諸行無常の響きあり。
沙羅双樹の花の色、盛者必衰の理をあらはす。
奢れる人も久からず、ただ春の夜の夢のごとし。
猛き者も遂にはほろびぬ、偏ひとへに風の前の塵におなじ。

 遠く異朝をとぶらへば、秦の趙高、漢の王莽、梁の朱异、唐の祿山、これらは皆舊主先皇の政にもしたがはず、樂しみをきはめ、諌めをも思ひ入れず、天下の亂れん事を悟らずして、民間の愁ふるところを知らざりしかば、久しからずして、亡じにし者どもなり。\\
近く本朝をうかがふに、承平の將門、天慶の純友、康和の義親、平治の信賴、これらはおごれる心もたけき事も、皆とりどりにこそありしかども、まぢかくは六波羅の入道、前太政大臣平朝臣清盛公と申しし人のありさま、傳へ承るこそ心もことばも及ばれね。
\end{tcolorbox}

次に \LaTeX での空白の取り扱いについて説明します.
ここの例では半角空白に関して説明します.
少々わかりにくいですが,【入力】では \verb|This| と \verb|is| の間に半角空白を一つ,\verb|is| と \verb|a| の間に半角空白を二つ,\verb|a| と \verb|pen.| の間に半角空白を三つ入れていますが,【出力】では無視されて一つ分の空白しか出てきません.

\begin{tcolorbox}[title={空白の処理}]
【入力(\verb|tex| ファイルの中身)】
\begin{verbatim}
This is  a   pen.
\end{verbatim}
\tcblower
【出力(\verb|pdf| ファイルでの見た目)】\\
This is a pen.
\end{tcolorbox}

逆に空白を(自分の好きなサイズで)出力したい場合は \verb|\hspace{長さ}| や \verb|\vspace{長さ}| といったコマンドを使用します.

\subsection{相互参照}
\label{ssec:ref}

\LaTeX で文書を書くメリットの一つに相互参照の機能が充実していることが挙げられます.
相互参照は「第~\ref{ssec:ref}~節を参照されたい.」や「式~\eqref{eq:NSr} を代入すると~」のような文脈において,対応する章・節・式・図・表などの番号を文書中から探し出す機能です.
また,ハイパーリンクを有効にしておくことで,\verb|pdf| ファイルに適切なリンクが埋め込まれ,クリックで該当箇所に飛ぶことができ大変便利です.
この学位論文テンプレートでもハイパーリンクを有効化しており,青字の文字列をクリックすると参照先に飛べます.
図表や章題目などを \verb|\label{}| コマンドでラベリングし,参照する文章中で \verb|\ref{}| コマンドを使って呼び出すのが基本的な相互参照の形です.
また,数式を参照する際は \verb|\eqref{}| コマンドの使用が一般的です.
図表や式をラベリングしておくことで,式番号や図番号が変わったとしてもその変化に合わせて \verb|pdf| ファイルの出力も変えられます.
また,論文を書き進める途中で章や節の順番が丸ごと入れ替わるといった事態が起きても,相互参照の機能を使っていれば全く問題ありません.
ただし,同じ名前のラベルは使用できないため,名前の重複には気をつけましょう.
存在しないラベル名を参照してもエラーとなり,\verb|pdf| ファイル中での出力は ?? となります.


\section{\LaTeX での数式の書き方}
\label{sec:formula_in_LaTeX}

\subsection{基本的な数式の記法}
\label{ssec:basic_command}

最も基本的な別行立ての数式は \verb|equation| 環境で記述します.
\begin{equation}
    \frac{\partial u_r}{\partial t} + (\mathbf{u}\cdot\nabla)u_r - \frac{u_\theta^2}{r} = -\frac{1}{\rho}\frac{\partial p}{\partial r} + \nu\left(\nabla^2 u_r - \frac{u_r}{r^2} - \frac{2}{r^2}\frac{\partial u_\theta}{\partial \theta}\right)
    \label{eq:NSr}
\end{equation}
式~\eqref{eq:NSr} の数式は \LaTeX でサポートされている最も標準的なコマンドで記述しています.
上付き添え字はキャレット \verb|^|,下付き添え字はアンダースコア \verb|_| で表現します.
したがって,$u_\theta^2$ は \verb|u_\theta^2| と書きます.
ここで注意点として,添え字が $u_\theta^2$ のように一文字であれば問題無いのですが,$R_{ij}$ のように二文字以上の場合は \verb|R_{ij}| のように括弧 \verb|{}| で囲んでください.
分数は \verb|\frac{分子}{分母}| で書き,常微分・偏微分の場合も同様です.
表~\ref{table:greek} のように $\rho$ や $\theta$ のといったギリシャ文字も出力できるほか,$\nabla$ や $\sin$,$\log$ のような数学で使う関数の類もコマンドが存在しています(例:\verb|\nabla|, \verb|\sin|, \verb|\log|).
$\sin$ や $\log$ は通常アップライト体(立体,Roman 体)で書きます.
$sin x$ などと書くことのないように気をつけましょう.
また,数式中で \LaTeX コマンドを使わずに変換で出した全角のθを入れている人をときどき見かけるので気をつけましょう.
式~\eqref{eq:NSr} の左辺と右辺で丸括弧の大きさが異なることにも注目してください.
左辺は特に何もしていませんが,右辺は分数がある分,縦方向に括弧の長さが必要です.
括弧の大きさを自動で調整するコマンドとして,\verb|\left(|,\verb|\right)| コマンドがあります.
左辺第二項では速度 $\mathbf{u}$ がベクトルであるため,\textbf{Bold} 体になっています.
これは \verb|\mathbf{u}| とすることで出力できます.
このテンプレートでは \verb|unicode-math| を使用しているので \verb|\symbf{u}| というコマンドもサポートされています.
ベクトルをボールドイタリック体 $\symbfit{u}$ にしたい場合は \verb|$\symbfit{u}$| というコマンドを使用してください(第~\ref{ssec:typeface_math}~節を参照).
このテンプレートでは \verb|bm| パッケージを使用していません.

式~\eqref{eq:NSr} は別行立ての数式でしたが,$E = mc^2$ のように文章中に組み込む数式(インライン数式)を出力する場合は \verb|$$| で数式を囲って \verb|$E = mc^2$| と書きます.

複数の数式を並べる場合は \verb|align| 環境を使いましょう.
\begin{align}
    a^2 &= b^2 + c^2 - 2bc\cos A \label{eq:cosA} \\
    b^2 &= c^2 + a^2 - 2ca\cos B \label{eq:cosB} \\
    c^2 &= a^2 + b^2 - 2ab\cos C \label{eq:cosC}
\end{align}
\verb|align| 環境では \verb|&| の位置で揃えるように制御しています.
式~\eqref{eq:cosA}--\eqref{eq:cosC} は $=$ の前に \verb|&| を置いているので,3 式の位置は $=$ の直前で揃っていることになります.
\verb|align| 環境を使えば
\begin{align}
    \sin 2\alpha &= \sin(\alpha + \alpha) \notag \\
    &= \sin\alpha\cos\alpha + \cos\alpha\sin\alpha \notag \\
    &= 2\sin\alpha\cos\alpha
    \label{eq:double-sin}
\end{align}
のように途中式も入れられます.
式番号を振らなくていい行は \verb|\notag| コマンドを使用します.
\begin{align}
    \varepsilon\left.\ab(\pdv[order=2]{\tilde{\phi}_1}{t} + g\pdv{\tilde{\phi}_1}{z})\right|_{z=0} &+ \varepsilon^2\left[\pdv[order=2]{\tilde{\phi}_2}{t} + g\pdv{\tilde{\phi}_2}{z} + \tilde{\eta}_1\pdv{}{z}\ab(\pdv[order=2]{\tilde{\phi}_1}{t} + g\pdv{\tilde{\phi}_1}{z})\right. \notag \\
    &\left.\left.+ \pdv{}{t}\ab\{\ab(\pdv{\tilde{\phi}_1}{x}) + \ab(\pdv{\tilde{\phi}_1}{z})^2\}\right]\right|_{z=0} = 0
    \label{eq:multiple-lines}
\end{align}
式~\eqref{eq:multiple-lines} は式の途中で改行を挟む場合の処理方法を示しています.
ただしソースコードは一部,第~\ref{ssec:physics2}~節で紹介する \verb|physics2| パッケージを使用しているので注意してください.
式~\eqref{eq:multiple-lines} で注目すべき箇所は大括弧 $[\quad]$ の途中で改行を挟んでいる点,代入記法として $|_{z=0}$ を使用している点の二箇所です.
一つの行内で \verb|\left| コマンドに対応する \verb|\right| コマンドが無いとエラーとなります.
大括弧 $[\quad]$ 内で改行する際は一行目で \verb|\left[| に対して \verb|\right.| を使用し,二行目で \verb|\right]| に対して \verb|\left.| を使用することで対処しています.
\verb|\left.| と \verb|\right.| は何も出力されません.
\verb|\left| または \verb|\right| に対応するものを用意するために使用します.
同様に,代入記法としての $|_{z=0}$ は \verb"\right|_{z=0}" に対して \verb|\left.| を使用しています.

場合分けのある数式は \verb|cases| 環境が便利です.
式~\eqref{eq:Kronecker} は Kronecker のデルタです.
\begin{equation}
    \delta_{ij} = 
    \begin{cases}
        1, & i = j \\
        0, & i \neq j
    \end{cases}
    \label{eq:Kronecker}
\end{equation}

\begin{table}[tp]
    \centering
    \caption{ギリシャ文字の出力方法.}
    \label{table:greek}
    \begin{tabular}{c|c|c|c|c} \hline\hline
        英語名 & 大文字出力 & 大文字入力 & 小文字出力 & 小文字入力 \\ \hline
        alpha & $A$ & \verb|A| & $\alpha$ & \verb|\alpha| \\ \hline
        beta & $B$ & \verb|B| & $\beta$ & \verb|\beta| \\ \hline
        gamma & $\Gamma$, $\varGamma$ & \verb|\Gamma|, \verb|\varGamma| & $\gamma$ & \verb|\gamma| \\ \hline
        delta & $\Delta$, $\varDelta$ & \verb|\Delta|, \verb|\varDelta| & $\delta$ & \verb|\delta| \\ \hline
        epsilon & $E$ & \verb|E| & $\epsilon$, $\varepsilon$ & \verb|\epsilon|, \verb|\varepsilon| \\ \hline
        zeta & $Z$ & \verb|Z| & $\zeta$ & \verb|\zeta| \\ \hline
        eta & $\Eta$ & \verb|\Eta| & $\eta$ & \verb|\eta| \\ \hline
        theta & $\Theta$, $\varTheta$ & \verb|\Theta|, \verb|\varTheta| & $\theta$, $\vartheta$ & \verb|\theta|, \verb|\vartheta| \\ \hline
        iota & $I$ & \verb|I| & $\iota$ & \verb|\iota| \\ \hline
        kappa & $K$ & \verb|K| & $\kappa$, $\varkappa$ & \verb|\kappa|, \verb|\varkappa| \\ \hline
        lambda & $\Lambda$, $\varLambda$ & \verb|\Lambda|, \verb|\varLambda| & $\lambda$ & \verb|\lambda| \\ \hline
        mu & $M$ & \verb|M| & $\mu$ & \verb|\mu| \\ \hline
        nu & $N$ & \verb|N| & $\nu$ & \verb|\nu| \\ \hline
        xi & $\Xi$, $\varXi$ & \verb|\Xi|, \verb|\varXi| & $\xi$ & \verb|\xi| \\ \hline
        omicron & $O$ & \verb|O| & $o$ & \verb|o| \\ \hline
        pi & $\Pi$, $\varPi$ & \verb|\Pi|, \verb|\varPi| & $\pi$, $\varpi$ & \verb|\pi|, \verb|\varpi| \\ \hline
        rho & $P$ & \verb|P| & $\rho$, $\varrho$ & \verb|\rho|, \verb|\varrho| \\ \hline
        sigma & $\Sigma$, $\varSigma$ & \verb|\Sigma|, \verb|\varSigma| & $\sigma$, $\varsigma$ & \verb|\sigma|, \verb|\varsigma| \\ \hline
        tau & $T$ & \verb|T| & $\tau$ & \verb|\tau| \\ \hline
        upsilon & $\Upsilon$, $\varUpsilon$ & \verb|\Upsilon|, \verb|\varUpsilon| & $\upsilon$ & \verb|\upsilon| \\ \hline
        phi & $\Phi$, $\varPhi$ & \verb|\Phi|, \verb|\varPhi| & $\phi$, $\varphi$ & \verb|\phi|, \verb|\varphi| \\ \hline
        chi & $X$ & \verb|X| & $\chi$ & \verb|\chi| \\ \hline
        psi & $\Psi$, $\varPsi$ & \verb|\Psi|, \verb|\varPsi| & $\psi$ & \verb|\psi| \\ \hline
        omega & $\Omega$, $\varOmega$ & \verb|\Omega|, \verb|\varOmega| & $\omega$ & \verb|\omega| \\ \hline
    \end{tabular}
\end{table}

\subsection{\texttt{physics2} パッケージ}
\label{ssec:physics2}

式~\eqref{eq:NSr} は基本的なコマンドを紹介するためにわざと面倒な書き方をしました.
しかし,複雑な数式は書くのも大変ですしミスの元になります.
また,数式のコードが長くなるとエラーの原因を探すのも難しくなります.
できるだけ数式を簡単に書くためのツールとして \verb|physics| パッケージが開発され,現在まで広く使われています.
ただ,\verb|physics| パッケージはいくつかの問題点を抱えており,最近は \verb|physics| パッケージの代替となる \verb|physics2| パッケージが開発されています.
\verb|physics2| パッケージは現在も開発途上のパッケージであるため,使い方は公式ドキュメントをよく読んでください.
\verb|physics2| パッケージでは常微分・偏微分用のコマンドはサポートされていないので,このテンプレートでは \verb|fixdif| パッケージと \verb|derivative| パッケージを併せて使用しています.
\begin{equation}
    \pdv{u_r}{t} + (\mathbf{u}\cdot\nabla)u_r - \frac{u_\theta^2}{r} = -\frac{1}{\rho}\pdv{p}{r} + \nu\ab(\nabla^2 u_r - \frac{u_r}{r^2} - \frac{2}{r^2}\pdv{u_\theta}{\theta})
    \label{eq:NSr2}
\end{equation}
式~\eqref{eq:NSr2} は式~\eqref{eq:NSr} を \verb|physics2| パッケージなどで書き換えたものです.
偏微分と大きさ調整が必要な括弧の記述が幾分か楽になりましたね.
また,微積分で使用する $\d$ は物理量の $d$ と区別するために直立体で表記することが推奨されています.
\verb|fixdif| パッケージのコマンドで $\d x$ と出力するには \verb|\d x| と入力します.
また,常微分の記法としては \verb|derivative| パッケージの \verb|\odv| コマンドを使用します.

\subsection{\texttt{siunitx} パッケージ}
\label{ssec:siunitx}

数式中の物理量は \textit{Italic} 体で表記しますが,単位は直立体で表記するのが一般的です.
また,数値と単位の間には空白を設けるのが一般的な書き方です\footnote{例外的に空白を設けなくてもいい単位として,角度を表す $\si{\degree}$ があります.$\ang{45}$ のように数値と単位を詰めて書くことが許容されています.ただし,温度を表す $\si{\degreeCelsius}$ は空白が必要です($\SI{45}{\degreeCelsius}$).}.
これらの要求を満たして簡単に単位を書けるのが \verb|siunitx| パッケージです.
\verb|siunitx| パッケージでは大きく分けてテキストモードとマクロモードの二種類の書き方があります.
テキストモードでは \verb|\si{W/m^2 K}| の数式のように書けるのに対し,マクロモードでは \verb|\si{\watt/\meter^2\kelvin}| のようにそれぞれの単位で用意されているコマンドで出力します.
出力結果はどちらも $\si{W/m^2 K}$ です.
$\si{\degreeCelsius}$ のようにマクロモードでしか使用できない単位もあります.
\verb|siunitx| パッケージは非常に便利なので以下のコマンドを積極的に使用しましょう.
単位のみの出力は \verb|\si{}| コマンド,数値と単位を併せての出力は \verb|\SI{}| コマンドを使用します.
\verb|\SI{}| コマンドを使用すると,数値と単位の間に適切な長さの空白を自動で入れてくれます.

\begin{tcolorbox}[title={\texttt{siunitx} パッケージ}]
    \begin{tabular}{lll}
        【テキストモードとマクロモードと混在】\\
        \textgt{コマンド}  & \textgt{出力} \\ \hline
        \verb|\si{W/m^2 K}|   & $\si{W/m^2 K}$ \\
        \verb|\si[per-mode=symbol]{\watt\per{\square\meter\kelvin}}|  & $\si[per-mode=symbol]{\watt\per{\square\meter\kelvin}}$ \\
        \verb|\si{\watt/\meter^2\kelvin}|  & $\si{\watt/\meter^2\kelvin}$
    \end{tabular}
    \tcblower
    \begin{tabular}{ll}
        【さまざまな単位の出力】\\
        \textgt{コマンド}  & \textgt{出力} \\ \hline
        \verb|\SI{45}{W/m^2 K}|   & $\SI{45}{W/m^2 K}$ \\
        \verb|\SI{45}{\degreeCelsius}|  & $\SI{45}{\degreeCelsius}$ \\
        \verb|\SI{45}{\micro\meter}|    & $\SI{45}{\micro\meter}$ \\
        \verb|\SI{45}{\um}|    & $\SI{45}{\um}$ \\
        \verb|\SI{45}{mL}|    & $\SI{45}{mL}$ \\
    \end{tabular}
\end{tcolorbox}

最後に示したミリリットル $\si{mL}$ には気をつけてください.
リットル $\si{L}$ を昔は $\ell$ と表記したこともありましたが,「単位は直立体」という原則に合わないのでやめましょう.
また,小文字の $\si{l}$ だと数字の $1$ と紛らわしいので,人名由来の単位ではありませんがリットルは大文字 $\si{L}$ で書くようにしましょう.

\begin{tcolorbox}[title={第~\ref{ch:basic}~章の参考文献}, colback=yellow!5!white, colframe=yellow!75!black, coltitle=black]
    \begin{itemize}
        \item 奥村晴彦, 黒木裕介,[改訂第 9 版]\LaTeX 美文書作成入門, 技術評論社 (2023), pp.~25--40, 73--115.
        \item \href{https://qiita.com/Yarakashi_Kikohshi/items/131e2324f401c3effb84}{Qiita: 脱 physics パッケージして physics2 パッケージを使おう}
        \item \href{https://qiita.com/gawara-t/items/57834e06f7fd95c18d26}{Qitta:【LaTeX】physicsパッケージとphysics2パッケージ(+α)との対応一覧}
        \item \href{http://www.yamamo10.jp/yamamoto/comp/latex/make_doc/unit/index.php}{Yamamot's Laboratory: LaTeX 数値・単位(siunitx)}
    \end{itemize}
\end{tcolorbox}




%%% 図表の配置 %%%
\input{chapter/figure_table.tex}

%%% 表記に関するあれこれ %%%
\chapter{表記に関するあれこれ}
\label{ch:notation}

\section{書体}
\label{sec:typeface}

\LaTeX では文書中で書体を変える命令が豊富に用意されています.
ここで,通常の文章中で書体を変える命令と数式環境中で書体を変える命令は異なるという点に注意してください.
よく \LaTeX 初心者が混同して使用しているケースを見かけるので気をつけましょう.
また,この節で述べる書体変更の命令は全体のごく一部を抜粋したものです.
実際にはもっと多くの命令がありますが,機械系学生にとって使用頻度が高いものを選んで載せています.

\subsection{通常の文章中で書体を変える方法}
\label{ssec:typeface_normal}

通常の文章中で書体を変える方法の一部を下に示します.
何もしなければ本文は直立体で表示されます.
強調する場合は \textbf{Boldface} 体や \textit{Italic} 体に変更します.
これらは本来欧文書体に対して適用する命令です.
和文書体に関してはここでは \textgt{ゴシック} 体を紹介します.
ただし,普通に学位論文を書いている分には文章中で書体を変える場面はあまりないと思います.

\begin{tcolorbox}[title={通常の文章中で書体を変える方法}]
    \begin{tabular}{lll}
        \textgt{書体クラス}  & \textgt{コマンド}  & \textgt{出力} \\ \hline
        直立(roman, upright)  & \verb|\textrm{Roman}|   & \textrm{Roman} \\
        ボールド(boldface)    & \verb|\textbf{Boldface}|  & \textbf{Boldface} \\
        イタリック(italic)    & \verb|\textit{Italic}|  & \textit{Italic} \\
        サンセリフ(sans-serif) & \verb|\textsf{Sans-serif}|   & \textsf{Sans-serif} \\
        タイプライター(typewriter)    & \verb|\texttt{Typewriter}|    & \texttt{Typewriter} \\
        ゴシック(gothic)  & \verb|\textgt{ゴシック}|  & \textgt{ゴシック}
    \end{tabular}
\end{tcolorbox}

\subsection{数式環境中で書体を変える方法}
\label{ssec:typeface_math}

次に数式環境中で書体を変える方法の一部を下に示します.
普通に学位論文を書いている場合であっても数式環境中で書体を変える場面はそれなりにあるはずです.
カリグラフィーは Hamiltonian $\mathcal{H}$ や Lagrangian $\mathcal{L}$ で使います.
また,ベクトルをボールドイタリック体で表記することがありますが,\LaTeX 標準ではボールドイタリック体がサポートされていません.
そのため,レガシー \LaTeX では \verb|bm| パッケージという外部のパッケージをわざわざ読み込んで \verb|\bm{abcABC123}| とする必要がありました.
この学位論文テンプレートではモダン \LaTeX で使われる \verb|unicode-math| を使用しており,\verb|unicode-math| では従来の \verb|\math..| というコマンド以外に \verb|\sym..| というコマンドが用意されています.
ボールドイタリック体は \verb|unicode-math| に標準で存在し,\verb|\symbfit{}| コマンドを使用すると出力できます.
ただし,\verb|unicode-math| に対応していない \LaTeX テンプレートはまだまだ多く存在しているので,そのようなテンプレートを使用する際は自分で \verb|bm| パッケージを読み込む必要があります.

\begin{tcolorbox}[title={数式環境中で書体を変える方法}]
    \begin{tabular}{lll}
        \textgt{書体クラス}  & \textgt{コマンド}  & \textgt{出力} \\ \hline
        直立(roman, upright)  & \verb|\mathrm{abcABC123}|   & $\mathrm{abcABC123}$ \\
        ボールド(boldface)    & \verb|\mathbf{abcABC123}|  & $\mathbf{abcABC123}$ \\
        イタリック(italic)    & \verb|\mathit{abcABC123}|  & $\mathit{abcABC123}$ \\
        カリグラフィー(calligraphy)   & \verb|\mathcal{ABCDEFG}|    & $\mathcal{ABCDEFG}$ \\
        ボールドイタリック(bold italic)   & \verb|\symbfit{abcABC123}| & $\symbfit{abcABC123}$
    \end{tabular}
\end{tcolorbox}

基本的に数式環境中のアルファベットは \textit{Italic} 体で,数字は直立体で表示されます.
日本機械学会の規定では無次元数を含め物理量は全て \textit{Italic} 体で記述するように決められているので,物理量自体を直立体にすることはありません.
ただし,添え字などで直立体にすることは考えられます.
飽和温度を $T_\mathrm{sat}$ と表記する場合には \verb|$T_\mathrm{sat}$| と入力します.

数式中のアルファベットは \textit{Italic} 体なのになぜ \verb|\mathit{}| コマンドが用意されているのでしょうか.

\begin{tcolorbox}
    \begin{tabular}{llll}
        \textgt{コマンド}  & \textgt{出力}  & \textgt{コマンド}  & \textgt{出力} \\ \hline
        \verb|$diff$|   & $diff$    & \verb|$\mathit{diff}$| & $\mathit{diff}$ \\
        \verb|$II$|   & $II$    & \verb|$\mathit{II}$| & $\mathit{II}$
    \end{tabular}
\end{tcolorbox}

このように,\verb|$diff$| とするとそれぞれの文字が別の変数として扱われてしまいます.
\verb|$II$| の場合も同様です.
そのため,複数文字から成る変数は \verb|\mathit{}| で全て指定してあげるのが理想です.
新しいコマンドとして \verb|sty| ファイル内に \verb|\newcommand{\diff}{\mathit{diff}}| のように定義してあげると \verb|\diff| と打てばいいだけなので楽です.

\section{記号の用法}
\label{sec:mark}

記号を適切に使用していない学位論文をよく見かけるので,学術論文等での使用が想定される各種記号の使用方法を記載します.
使い方は研究分野やその人の思想,入力環境などにも依存するので明確な規則はありませんが,大まかな目安として考えてください.

\subsection*{横棒}

\begin{itemize}
    \item ハイフン(hyphen, -)
    \begin{itemize}
        \item Unicode: U+002D(厳密にはハイフンマイナス)
        \item \LaTeX での入力:\verb|-|
        \item 一般的な JIS キーボードであれば「ほ」や「=」があるキーを押すと出てくる.
        \item 英語などで見られる複合的な単語(例:large-scale structure).
        \item 大小関係の無い数字の接続(例:郵便番号など,278-8510).
        \item 厳密にはハイフン(hyphen, U+2010)ではなくハイフンマイナス(hyphen-minus).一般的なキーボードから入力できるものはハイフンとマイナスの両方に用いられることがあるハイフンマイナスである.
    \end{itemize}
    \item en ダッシュ(en dash, --)
    \begin{itemize}
        \item Unicode: U+2013
        \item \LaTeX での入力:\verb|--|(ハイフン二つ)
        \item 複数の人物等を繋ぐ場合(例:Navier--Stokes equation).
        \item 大小関係のある数字を繋ぐ場合(例:Figs.~2--4, $\Re = 150$--$180$).日本語の文章では「~」の使用(Figs.~2~4, $\Re = 150$~$180$)をよく見るが,科学的な文章では不適切.en ダッシュを使うように.また,数式環境中で \verb|$--$| と入力するとマイナスが二つ出力されてしまう($--$)ので,一度数式環境を抜けて \verb|$150$--$180$| とするか \verb|$150\text{--}180$| とすること.
        \item 図の軸ラベルどうし($x$--$y$ 平面).\verb|$x-y$| としている例をよく見るがこれはマイナスとして処理されるので見た目が $x-y$ となってしまう.
    \end{itemize}
    \item em ダッシュ(em dash, ---)
    \begin{itemize}
        \item Unicode: U+2014
        \item \LaTeX での入力:\verb|---|(ハイフン三つ)
        \item 欧文中で文の区切りなどに用いる.理科系の文章ではあまり使用しない.
        \item 欧文中で説明や副題を設ける場合に使用.
    \end{itemize}
    \item 水平バー(horizontal bar, ―)
    \begin{itemize}
        \item Unicode: U+2015
        \item \LaTeX での入力:\verb|―|(直接入力する場合),\verb|\symbol{"2015}|(Unicode で指定して入力する場合)
        \item 和文中で説明や副題を設ける場合に二つ並べ,倍角ダッシュとして使用する.環境によっては間に空白が入ってしまうので工夫が必要(例:「〇〇に関する研究 \――△△の観点から\――」).このテンプレートでは \verb|\――| で空白を作らずに出力できる.
    \end{itemize}
    \item マイナス(minus, $-$)
    \begin{itemize}
        \item Unicode: U+2212
        \item \LaTeX での入力:\verb|$-$|
        \item 数式で減算・差を表す際に使用.数式環境に入れ忘れてハイフンで出力されているケースをよく見るので注意.
    \end{itemize}
\end{itemize}

\subsection*{引用符}
\label{ssec:quotation}

例えば次のような文を出力したい場合どのように入力すればいいでしょうか.
\begin{quotation}
    ``I'm a student!'' \qquad `I'm a student!'
\end{quotation}
正解は
\begin{quotation}
    \verb|``I'm a student!''| \qquad \verb|`I'm a student!'|
\end{quotation}
です.
引用や強調の際に使用する引用符を \LaTeX で出力する際は,左引用符を \verb|`| で\footnote{\texttt{`} は普段あまり使用しないキーなのでどこにあるのか探し回る人が多いです.一般的な JIS キーボードであればアットマーク @ のキーにあります.\texttt{Shift + @} で出力されます.},右引用符を \verb|'| で囲います\footnote{I'm のアポストロフィと同じです.一般的な JIS キーボードであれば数字の 7 のキーにあります.\texttt{Shift + 7} で出力されます.}.
単一引用符 \verb|`'| の場合も二重引用符 \verb|``''| の場合も同様です.
キーボードにデフォルトで搭載されている二重引用符 \verb|"| で囲って \verb|"I'm a student!"| のようにはしないので注意してください.
仮に
\begin{quotation}
    \verb|"I'm a student!"| \qquad \verb|'I'm a student!'|
\end{quotation}
と入力したらどのように出力されるでしょうか.
\begin{quotation}
    "I'm a student!" \qquad 'I'm a student!'
\end{quotation}
このようになりました.
明らかに変ですよね.
気をつけましょう.
また,日本機械学会の規定では文献リストの各文献タイトルを引用符で括ることはしないので問題ありませんが,海外ジャーナルなどでは文献タイトルを二重引用符で括るのはよく使われる方法なので覚えておきましょう.

\section{特殊な文字を含む固有名詞などを出力する方法}
\label{sec:unicode}

本節の内容は特に謝辞や文献リストの作成など,人名を多く書く際に役立つと思います.
例えば「あれ,Schrödinger ってどうやって出せばいいんだ?」みたいな経験をされた方は多いのではないでしょうか.
また,日本人の人名にはさまざまな漢字が使われており,中には通常の変換機能ではなかなか出せない漢字を使用している人もいます.

このテンプレートでは \LuaLaTeX を採用しているため,デフォルトで UTF-8 がサポートされています.
したがって,特殊な文字を直接 \verb|tex| ファイルに入力してもエラーにならずにそのまま出力できます.
例えば先程の Schrödinger の場合は,どこか別のところから文字だけコピーしてきて直接 \verb|Schrödinger| と打てばそのまま出力できます.
コピーしてくる先が無い場合は \pLaTeX などレガシー \LaTeX で使用するコマンドで \verb|Schr\"{o}dinger| とすることで出力できます.
欧文で使用するアクセント類は \LaTeX の入門書には大抵記載があるので詳細はそちらを読んでください.

日本語の場合でも同様で,Unicode でサポートされている文字であればそのまま出力できます.
例えば,いわゆる梯子高の「髙橋」さんや「飛驒山脈」などは問題無く出力できます.
「髙」や「驒」は環境によっては文字化けの原因になるので,これは UTF-8 がサポートされている大きなメリットですね.
ただ,こちらもコピーしてくる先があればいいのですが,無い場合は困ります.
その際は Unicode を指定して文字を出力するようにしましょう\footnote{異体字などの Unicode を調べるには \href{https://glyphwiki.org/wiki/}{グリフウィキ} を使用すると便利です.}.
例えば「飛\symbol{"9A52}山脈」の場合は「\verb|飛\symbol{"9A52}山脈|」とすれば出力できます.
「驒」の Unicode は U+9A52 です.
Unicode の U+ 以降の文字列を \verb|\symbol{"xxxx}| の \verb|xxxx| の箇所に入れれば大丈夫です.
ただし,「\CID{1481}」と「\CID{7652}」など一部例外として別の字体に同じ Unicode が割り当てられていることがあるので,その際は OpenType の CID 番号を使って \verb|\CID{1481}|, \verb|\CID{7652}| のように書きます\footnote{「\CID{1481}」と「\CID{7652}」の場合はどちらも U+845B です.CID 番号をグリフウィキで調べる際は aj1-01481 など,aj1 を目印に探してください.aj1 は Adobe-Japan1 の略称です.}.

\begin{tcolorbox}[title={第~\ref{ch:notation}~章の参考文献}, colback=yellow!5!white, colframe=yellow!75!black, coltitle=black]
    \begin{itemize}
        \item 奥村晴彦, 黒木裕介,[改訂第 9 版]\LaTeX 美文書作成入門, 技術評論社 (2023), pp.~41--44, 92--95, 248--253.
        \item \href{https://glyphwiki.org/wiki/}{グリフウィキ}
    \end{itemize}
\end{tcolorbox}




%%% BibTeX による参考文献一覧の出力 %%%
\input{chapter/bibtex.tex}

%%% 先生や先輩に添削してもらうときの注意点 %%%
\input{chapter/check.tex}

%%% さらに詳しい情報が欲しい人は %%%
\input{chapter/information.tex}

%%% 謝辞 %%%
\input{chapter/acknowledgement.tex}

%%% 文献 %%%
\biblist
% \nocite{*} が有効のとき,引用していない文献も含めて全て表示
% 確認用なので論文提出前には必ず \nocite{*} をコメントアウトすること
\nocite{*}
% 使用する bst ファイル
\bibliographystyle{jsme}
% 読み込む bib ファイル
\bibliography{
    mybib_en.bib,
    mybib_jp.bib
}

%%% 付録 %%%
\input{chapter/appendix.tex}

%%%%%%%%%%%%%%%%%%%%%%%%%%%%%%
%%% 文章を書けるのはここまで %%%
%%%%%%%%%%%%%%%%%%%%%%%%%%%%%%
\end{document}
